\documentclass[a4paper,12pt]{article}
\usepackage{multicol}
\usepackage{calc}
\usepackage{ifthen}
\usepackage{hyperref}
\usepackage{multicol}
\usepackage{calc}
\usepackage{ifthen}
\usepackage[portrait]{geometry}
\usepackage{hyperref}
\usepackage{amsmath}
\usepackage{amssymb}
\usepackage{xcolor}
\usepackage{colortbl}
\usepackage{array}
\usepackage{enumitem}

% To make this come out properly in landscape mode, do one of the following
% 1.
%  pdflatex latexsheet.tex
%
% 2.
%  latex latexsheet.tex
%  dvips -P pdf  -t landscape latexsheet.dvi
%  ps2pdf latexsheet.ps


% If you're reading this, be prepared for confusion.  Making this was
% a learning experience for me, and it shows.  Much of the placement
% was hacked in; if you make it better, let me know...


% 2008-04
% Changed page margin code to use the geometry package. Also added code for
% conditional page margins, depending on paper size. Thanks to Uwe Ziegenhagen
% for the suggestions.

% 2006-08
% Made changes based on suggestions from Gene Cooperman. <gene at ccs.neu.edu>


% To Do:
% \listoffigures \listoftables
% \setcounter{secnumdepth}{0}


% This sets page margins to .5 inch if using letter paper, and to 1cm
% if using A4 paper. (This probably isn't strictly necessary.)
% If using another size paper, use default 1cm margins.
\ifthenelse{\lengthtest { \paperwidth = 11in}}
	{ \geometry{top=.5in,left=.5in,right=.5in,bottom=.5in} }
	{\ifthenelse{ \lengthtest{ \paperwidth = 297mm}}
		{\geometry{top=1cm,left=1cm,right=1cm,bottom=1cm} }
		{\geometry{top=1cm,left=1cm,right=1cm,bottom=1cm} }
	}

% Turn off header and footer
\pagestyle{empty}
 

% Redefine section commands to use less space
\makeatletter
\renewcommand{\section}{\@startsection{section}{1}{0mm}%
                                {-1ex plus -.5ex minus -.2ex}%
                                {0.5ex plus .2ex}%x
                                {\normalfont\large\bfseries}}
\renewcommand{\subsection}{\@startsection{subsection}{2}{0mm}%
                                {-1explus -.5ex minus -.2ex}%
                                {0.5ex plus .2ex}%
                                {\normalfont\normalsize\bfseries}}
\renewcommand{\subsubsection}{\@startsection{subsubsection}{3}{0mm}%
                                {-1ex plus -.5ex minus -.2ex}%
                                {1ex plus .2ex}%
                                {\normalfont\small\bfseries}}


% Don't print section numbers
\setcounter{secnumdepth}{0}


\setlength{\parindent}{3pt}
\setlength{\parskip}{0pt plus 0.2ex}

\begin{document}

\renewcommand{\arraystretch}{2}




% multicol parameters
% These lengths are set only within the two main columns
%\setlength{\columnseprule}{0.25pt}
\setlength{\premulticols}{5pt}
\setlength{\postmulticols}{5pt}
\setlength{\multicolsep}{5pt}
\setlength{\columnsep}{6pt}

\section{Derivatives}
\begin{tabular}{lll}
Linearity & 
$\displaystyle {\frac {d}{{{d}}x}}(\alpha f +\beta g )=\alpha f'+\beta g'$  \\
Product & $(fg)'={fg'}+{f'g}$ & \\
(7.17) & $(f^{n})'(x)=nf^{n-1}(x)\cdot f'(x)$ & \\
Power & 
\begin{tabular}{l}
$(x^n)'=nx^{n-1}$ (integer $n\neq 0$) \\
$(x^r)'=rx^{r-1}$ ($r\in\mathbb{R}$, $x>0$)
\end{tabular} & \\
Quotient & $\displaystyle\left( \frac{f}{g} \right)'(x)=\frac{f'g-fg'}{g^2}$ & $g(x)\neq 0$ \\
Reciprocal & 
\begin{tabular}{l}
$\displaystyle\left( \frac{1}{f} \right)'=\frac{{-f'}}{f^2}$ \\
$\displaystyle\frac{d}{dx}\left[ \frac{1}{f(x)} \right]=\frac{{-f'(x)}}{[f(x)]^2}$
\end{tabular} & $f(x)\neq 0$ \\
Chain Rule & 
\begin{tabular}{l}
$(f(g(x)))'=f'(g(x))\cdot g'(x)$ \\
$(f(g(h(x))))'=f'(g(h(x))) \cdot g'(h(x)) \cdot h'(x)$ \\
\end{tabular} & 
$\displaystyle {\frac {dz}{dx}}={\frac {dz}{dy}}\cdot {\frac {dy}{dx}}$ \\
Inverse & $\displaystyle \big(f^{−1}\big)'(x)=\frac{1}{f'\big(f^{−1}(x)\big)}$

\end{tabular}

\subsubsection{Common Derivatives}

\begin{tabular}{ll}

$\displaystyle {\color{gray}\frac{d}{dx} \left( {\color{black}{\sqrt{x}}}\right) } = \frac{1}{2\sqrt{x}}$ \\
$\displaystyle {\color{gray}\frac{d}{dx} \left( {\color{black}{\sqrt{g(x)}}}\right) } = \frac{g'(x)}{2\sqrt{g(x)}}$ \\
\hline
$\displaystyle {\color{gray}\frac{d}{dx} \left( {\color{black}{\log_a x}}\right) } = \frac{1}{x \ln a}$ \\
$\displaystyle {\color{gray}\frac{d}{dx} \left( {\color{black}{\ln x}}\right) } = \frac{1}{x}$ \\
$\displaystyle {\color{gray}\frac{d}{dx} \left( {\color{black}{a^x}}\right) } = a^x \ln a$ \\
$\displaystyle {\color{gray}\frac{d}{dx} \left( {\color{black}{e^x}}\right) } = e^x$ \\
\hline
$\displaystyle {\color{gray}\frac{d}{dx} \left( {\color{black}{\sin x}}\right) } = \cos x$ \\
$\displaystyle {\color{gray}\frac{d}{dx} \left( {\color{black}{\cos x}}\right) } = -\sin x$ \\
$\displaystyle {\color{gray}\frac{d}{dx} \left( {\color{black}{\tan x}}\right) } = \sec^2 x=1+\tan^2 x$ \\
$\displaystyle {\color{gray}\frac{d}{dx} \left( {\color{black}{\cot x}}\right) } = -\csc^2 x=-1-\cot^2 x$ \\
\hline
$\displaystyle {\color{gray}\frac{d}{dx} \left( {\color{black}{\arcsin x}}\right) } = \frac{1}{\sqrt{1-x^2}}$ \\
$\displaystyle {\color{gray}\frac{d}{dx} \left( {\color{black}{\arccos x}}\right) } = -\frac{1}{\sqrt{1-x^2}}$ \\
$\displaystyle {\color{gray}\frac{d}{dx} \left( {\color{black}{\arctan x}}\right) } = \frac{1}{1+x^2}$ \\
$\displaystyle {\color{gray}\frac{d}{dx} \left( {\color{black}{\text{arccot} x}}\right) } = -\frac{1}{1+x^2}$

\end{tabular}

\end{document}
